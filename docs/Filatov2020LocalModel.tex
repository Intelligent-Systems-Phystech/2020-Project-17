\documentclass[12pt, twoside]{article}
\usepackage{jmlda}
\newcommand{\hdir}{.}

\begin{document}

\title
    [Исследование свойств локальных моделей] % краткое название; не нужно, если полное название влезает в~колонтитул
    {Исследование свойств локальных моделей в задаче декодирования сигналов головного мозга}
\author
[] % список авторов (не более трех) для колонтитула; не нужен, если основной список влезает в колонтитул
{} % основной список авторов, выводимый в оглавление
[Филатов А.В., Маркин В.О.] % список авторов, выводимый в заголовок; не нужен, если он не отличается от основного
\email
{filatov.av@phystech.edu; markin.vo@phystech.edu}
\organization
{МФТИ}
\abstract
    {
		В данной работе рассматривается проблема создания нейрокомпьютерного интерфейса. Особенностью этой проблемы является требование устойчивости у моделей. При построении систем нейрокомпьютерного интерфейса используются линейные модели. Важным аспектом создания таких моделей является построение надежного
		% определение оптимальности (и заменить надежность)
		% заменить декодирование на multiway декодирование
		% стабильность (устойчивость)
		 признакового пространства. В данной статье рассматривается построение признакового пространства на данных электрокортикограммы для метода частичной регрессии наименьших квадратов
		
		Основной вклад данной работы заключается в учете пространственной зависимости в модели. В нашем подходе используется пространственная аппроксимация искомого сигнала с помощью нормального распределения и полиномов. Информация об аппроксимации учитывается при создании признакового пространства. В статье приведены результаты численных экспериментов на данных электрокортикограмм головного мозга обезъян.
		
\bigskip
\noindent
\textbf{Ключевые слова}: \emph {отбор признаков; нейрокомпьютерный интерфейс; электрокортикограмма; локальные модели }
}

\maketitle
\section{Введение}
Нейрокомпьютерный интерфейс (BCI) \cite{shih2012brain} 
считывает сигналы нейронов головного мозга, анализировать их   и переводить в команды исполняющей системы. Исследования в данной области позволяют людям с нарушениями двигательных функций организма заменить или восстановить их. Примером такой системы является система управления роботизированным протезом посредством мозговых импульсов. 

Мозговая активность представляет собой совокупность электрических импульсов различной амплитуды и частоты, возникающих на поверхности головного мозга. Исследование мозговой активности производится при помощи  электрокортикографии \cite{hill2012recording} или \cite{aminoff2012electroencephalography} электроэнцефалографии. В результате измерения мы получаем временной ряд напряжений сигнала, который используется как данные для задачи. В задаче используется данные из \cite{chao2010long}.

 Подходы \cite{morishita2014brain, alexander2013traveling} к решению задачи состоят в извлечении информативных признаков из пространственных, частотных и временных характеристик сигнала . 
В \cite{chin2007identification, eliseyev2014stable, loza2017unsupervised} исследуются частотные характеристики. Основными методами решения являются PLS \cite{eliseyev2014stable,eliseyev2016penalized, rosipal2005overview}, PCA \cite{rosipal2005overview, eliseyev2016penalized}. В \cite{zhao2014coupled} используются алгоритмы, построенные на скрытых марковских моделях. В \cite{loza2017unsupervised, zhao2010ecog} рассматриваются различные участки сигнала в виде слов. В работе \cite{motrenko2018multi} задача отбора признаков сводится к задаче квадратичного программирования (QuadraticProgramming Feature Selection \cite{rodriguez2010quadratic}). Также для решения задачи используются нейросетевые модели\cite{xie2018deep}. 
\newpage
\section{Постановка задачи}
Данные электрокортикограммы представляют собой временной ряд амплитуд сигналов $\mathbf{X}(t)  \in \RR^{m}$, по которым нужно предсказать положение запястья в следующим момент времени $\mathbf{y}(t+1) \in \RR^3$. В качестве выборки рассматривается $\mathfrak{D} = \{(\vec{X}_{(i)}^n, \mathbf{y_{i+1}}\}$, где $\vec{X}_{(i)}^n$ --- значения временнего ряда с момента времени $i$ по момент $i + n$, где $n$ --- гиперпараметер задачи и выбирается из дополнительных условий. В силу коррелированности исходных данных предлагается разбить предсказательную модель на локальную модель и модель регрессии.

\begin{Def}
	Локальная модель --- совокупность двух отображений: $\phi$ и $\psi$, где 
	$\phi$ отображает из пространства большей размерности в пространство меньшей размерности, а $\psi$ отображает из этого же пространства меньшей размерности в исходное пространство большей размерности, при чем при применение композиции $\psi \circ \phi$ 
\end{Def}

В нашем случае под количеством информации понимается дисперсия.
Оптимизационная задача на нахождение локальной модели ставится следующим образом
\[
	\phi: \RR^{n \times k_1} \rightarrow \RR^{n \times k_2}
\]
\[
	\psi: \RR^{n \times k_2} \rightarrow \RR^{n \times k_1}
\]
\[
	\psi^*, \phi^* = \argmin_{\psi, \phi} \|\vec{X} - \psi \circ \phi (\vec{X}) \|_2
\]

Локальная модели получаем новую выборку $\mathfrak{D}_{new} = \{(\vec{Z}_{(i)}^{n}, \vec{y}_i)\}$, $\vec{Z}_{(i)}^{n} = \phi(\vec{X}_{(i)}^{n})$. Эту задачу мы решаем при помощи регрессии:
\[
	\vec{w}^* = \argmin_{\vec{w}} L(\mathbf{z}, \vec{w}, \mathbf{y})
\]

Критерием качества линейной модели выступает коэффициент детерминации или $R^2$.


\section{Эксперимент}

Разбиение выборки производится в следующем соотношении: 80\% --- обучение, 20\% --- тестирование.

\newpage
\bibliographystyle{unsrt}
\bibliography{references}
\end{document}
