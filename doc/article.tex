\documentclass[12pt, twoside]{article}
\usepackage{jmlda}
\newcommand{\hdir}{.}

\begin{document}

\title
    [Прогнозирование намерений] % краткое название; не нужно, если полное название влезает в~колонтитул
    {Прогнозирование намерений. Исследование свойств локальных моделей при пространственном декодировании сигналов головного мозга}
\author
[] % список авторов (не более трех) для колонтитула; не нужен, если основной список влезает в колонтитул
{} % основной список авторов, выводимый в оглавление
[Филатов А.В., Маркин В.О.] % список авторов, выводимый в заголовок; не нужен, если он не отличается от основного
\email
{filatov.av@phystech.edu; markin.vo@phystech.edu}
\organization
{МФТИ}
\abstract
    {
		В данной работе рассматривается проблема создания нейрокомпьютерного интерфейса. Особенностью этой проблемы является требование стабильности у моделей. Поэтому при построении таких систем используются простые модели. Важным аспектом создания таких моделей является построение надежного признакового пространства. В данной статье рассматривается построение признакового пространства на данных электрокортикограммы для метода частичной регрессии наименьших квадратов
		
		Основной вклад данной работы заключается в учете пространственной зависимости в модели. В нашем подходе используется пространственная аппроксимация искомого сигнала с помощью нормального распределения и полиномов. Информация об аппроксимации учитывается при создании признакового пространства. В статье приведены результаты численных экспериментов на данных электрокортикограмм головного мозга обезъян.
		
\bigskip
\noindent
\textbf{Ключевые слова}: \emph {отбор признаков; нейрокомпьютерный интерфейс; электрокортикограмма; локальные модели }
}

\maketitle
\section{Введение}
Нейрокомпьютерный интерфейс (BCI) \cite{shih2012brain} позволяет считывать сигналы нейронов головного мозга, анализировать их   и переводить в команды исполняющей системы. Исследования в данной области позволяют людям с нарушениями двигательных функций организма заменить или восстановить их. Примером такой системы является система управления роботизированным протезом посредством мозговых импульсов. 

Мозговая активность представляет собой совокупность электрических импульсов различной амплитуды и частоты, возникающих в коре головного мозга. Исследование мозговой активности может производиться при помощи  электрокортикографии \cite{hill2012recording} или \cite{aminoff2012electroencephalography} электроэнцефалографии. На выходе этих процедур мы получаем временной ряд напряжений сигнала, который используется как сырые данные для задачи. В нашей задаче используется данные из \cite{chao2010long}.

Стандартные подходы к решению задачи состоят в извлечении информативных признаков из пространственных, частотных и временных характеристик сигнала \cite{morishita2014brain, alexander2013traveling}. 
Большинство методов в смежных работах исследуют частотные характеристики \cite{chin2007identification, eliseyev2014stable, loza2017unsupervised}. Наиболее распространёнными моделями являются алгоритмы PLS \cite{eliseyev2014stable,eliseyev2016penalized, rosipal2005overview}, PCA \cite{rosipal2005overview, eliseyev2016penalized}. В работе \cite{zhao2014coupled} используются алгоритмы, построенные на скрытых марковских моделях. В работах \cite{loza2017unsupervised, zhao2010ecog} авторы рассматривают различные участки сигнала в виде слов. В работе \cite{motrenko2018multi} задача отбора признаков сводится к задаче квадратичного программирования (QuadraticProgramming Feature Selection \cite{rodriguez2010quadratic}). Также для решения задачи используются нейросетевые модели\cite{xie2018deep}. 
\newpage
\section{Постановка задачи}
Данные электрокортикограммы представляют собой временной ряд амплитуд сигналов $\mathbf{X}(t_i)  \in \RR^{m}$, по которым нужно предсказать положение запястья в следующим момент времени $\mathbf{y}(t_{i+1}) \in \RR^3$. В качестве выборки рассматривается $\mathfrak{D} = \{(X_i^{i+n_{wind}}), \mathbf{y_{i+1}}\}$, где $X_i^{i+n_{wind}}$ --- значения временнего ряда с момента времени $i$ по момент $i + n_{wind}$, где $n_{wind}$ --- гиперпараметер задачи и выбирается из дополнительных условий. В силу коррелированности исходных данных предлагается разбить предсказательную модель на локальную модель и модель регрессии.

\begin{Def}
	Локальная модель --- совокупность двух отображений: $\phi$ и $\psi$, где 
	$\phi$ отображает из пространства большей размерности в пространство меньшей размерности, а $\psi$ отображает из этого же пространства меньшей размерности в исходное пространство большей размерности, при чем при применение композиции $\psi \circ \phi$ сохраняется максимальное количество информации
\end{Def}

В нашем случае под количеством информации понимается дисперсия.
Оптимизационная задача на нахождение локальной модели ставится следующим образом
\[
	\phi: \RR^{N_{wind} \times m} \rightarrow \RR^{N_{wind} \times n}
\]
\[
	\psi: \RR^{N_{wind} \times n} \rightarrow \RR^{N_{wind} \times m}
\]
\[
	\psi^*, \phi^* = \argmin_{\psi, \phi} \|\mathbf{S} - \psi \circ \phi (\mathbf{S}) \|_2
\]
После нахождения локальной модели получаем новую выборку $\mathfrak{D}_{new} = \{(z_{i}^{i+n_{wind}}, \mathbf{x}_i)\}$, $z_{i}^{i+n_{wind}} = \phi(S_{i}^{i + n_{wind}})$. Эту задачу мы решаем при помощи регрессии:
\[
	W^* = \argmin_W L(\mathbf{z}, \mathbf{W}, \mathbf{y})
\]

Критерием качества линейной модели выступает коэффициент детерминации или $R^2$, как основный критерий в задачах линейной регрессии.

Разбиение выборки производится в следующем соотношении: 80\% --- обучение, 20\% --- тестирование.

\newpage
\bibliographystyle{unsrt}
\bibliography{references}
\end{document}
